\documentclass[a4paper]{scrartcl}
\usepackage[utf8]{inputenc}
\usepackage{ngerman}
\usepackage{mathtools}
\usepackage{amssymb}
\usepackage{pdfpages}
\usepackage{mathtools}
\usepackage{listings}
\usepackage{tikz}
\usepackage{qtree}
\usepackage{hyperref}
\usetikzlibrary{arrows,automata}
\usetikzlibrary{shapes.multipart}

\title{Künstliche Intelligenz}
\date{SS 2015}
\author{Markus Klemm.net}

\lstdefinestyle{customc}{
  belowcaptionskip=1\baselineskip,
  breaklines=true,
  frame=L,
  xleftmargin=\parindent,
  language=C,
  showstringspaces=false,
  basicstyle=\footnotesize\ttfamily,
  keywordstyle=\bfseries\color{green!40!black},
  commentstyle=\itshape\color{purple!40!black},
  identifierstyle=\color{blue},
  stringstyle=\color{orange},
}

\begin{document}
\maketitle
\tableofcontents

%Inhalte
%Prädikatenlogik
%Prolog-Programmierung
%Logisches Schließen
%Computeralgebra
%Sprachverarbeitung
%Problemlösen durch Suche
%%A*-Suche
%Bayes'sche Netze

\section{Prädikatenlogik}
\subsection{Problem} Aussagenlogik ist wenig mächtig. Aussagen, die sich nicht formulieren lassen:
\begin{itemize}
\item Alle Vögel können fliegen.
\item Wenn X eine Katze ist, dann ist X ein Haustier.
\item Für jedes Land gibt es eine Hauptstadt.
\end{itemize}
In der Prädikatenlogik betrachten wir zunächst eine (unwichtige) Menge $U$ (Universum), die alle zu betrachtenden Objekte (Vögel, Katzen, Länder) enthält.
Davon betrachten wir Teilmengen, z.B. die Menge aller Vögel (also eine einstellige Relation) oder die Menge aller Verheirateten (also eine mehrstellige Relation).
\subsection{Definition}
Sei $V$ eine Menge von Variablen, $K$ eine Menge von Konstanten, $F$ eine Menge von Funktionssymbolen.
\begin{itemize}
\item Dann sind alle Variablen in $V$ und Konstanten in $K$, Terme.
\item Wenn $t_1,\dots,t_n$ Terme und $f$ ein $n$-stelliges Funktionssymbol sind, dann ist auch $f(t_1,\dots,t_n)$ ein Term.
\end{itemize}
\paragraph{Beispiel} $V=\{x,y,z\},K=\{a,b,c\},F=\{+,*\}$ Terme sind $x,y,a,x+a,(x+a)*y$.

\subsection{Definition} Sei eine Menge von Prädikatensymbolen gegeben. Die Formeln der Prädikatenlogik sind induktiv.

\begin{itemize}
\item Wenn $t_1,\dots,t_n$ Terme und $P$ ein Prädikatensymbol der Stelligkeit $n$ ist, dann ist $P(t_1,\dots,t_n)$ eine Formel.
\item Wenn $F,G$ Formeln sind, dann auch $F \wedge G, F\vee G, \neg F$.
\item Wenn $x$ eine Variable und $F$ eine Formel sind, dann auch $\forall x F$, sowie $\exists x F$.
\end{itemize}
\paragraph{Beispiel} 
\begin{itemize}
\item katze(reni)
\item $\forall x (\text{katze}(x) \rightarrow \text{haustier}(x)) $
Syntaxbaum dazu \\* \begin{tikzpicture} 
\tikzstyle{level 1} = [sibling distance = 50 mm]
\tikzstyle{level 2} = [sibling distance = 30 mm]
\node{$\forall x$}
    child{node{$\rightarrow$}
        child{node{$\text{katze}(x)$}}
        child{node{$\text{haustier}(x)$}}
        }
        ;
\end{tikzpicture}
\item $\forall x(\text{land}(x) \rightarrow \exists y \text{hauptstadt}(x,y))$ \\*
\begin{tikzpicture} 
\tikzstyle{level 1} = [sibling distance = 50 mm]
\tikzstyle{level 2} = [sibling distance = 30 mm]
\node{$\forall x$}
    child{node{$\rightarrow$}
        child{node{$\text{land}(x)$}}
        child{node{$\exists y$}
            child{node{$\text{hauptstadt}(x,y)$}}
            }
        }
    ;
\end{tikzpicture}
\end{itemize}

\subsection{Semantik(informal)} Den Prädikatensymbolen müssen Relationen zugeordnet werden. Z.B: $\text{land} = \{\text{deutschland},\text{frankreich},\text{spanien}\}$\\
Wahrheitswert einer Formel:
\begin{itemize}
\item Die Formel $P(t_1,\dots,t_n)$ ist wahr, wenn $(t_1,\dots,t_n) \in P$
\item Die Wahrheit von $F \wedge G, F \vee G, \neg F$ ist wie in der Aussagenlogik definiert.
\item $\exists xF$ ist wahr, wenn es ein $x \in U$ gibt, so dass $F$ für dieses $x$ wahr ist.
\item $\forall xF$ ist wahr, wenn für alle $x \in U \quad F$ für diese $x$ wahr ist.
\end{itemize}
\paragraph{Beispiel}
\subparagraph{Symbole:} vogel,fliegt
\subparagraph{Relationen} $\text{vogel} = \{\text{amsel},\text{drossel},\text{fink},\text{star}\}, \text{fliegt} = \text{vogel} \cup \{\text{maikäfer},\text{A380}\}$
\begin{itemize}
\item $\text{vogel}(\text{amsel})$ ist wahr
\item $\exists x \; \text{vogel}(x)$ ist wahr
\item $\forall x \; \text{vogel}(x)$ ist falsch
\item $\forall x \; (\text{vogel}(x) \rightarrow \text{fliegt} (x))$ ist wahr
\end{itemize}

Vorrang der Operatoren:
\[ \begin{array}{c}
\neg \\
\forall \exists\\
\wedge, \vee\\
\rightarrow  , \leftrightarrow\\
\end{array}
\]

\paragraph{Beispiel} $\text{land} = \{\text{deutschland},\text{england},\text{frankreich},\text{spanien}\},$\\ $ \text{hauptstadt} = \{(\text{deutschland},\text{berlin}),(\text{england},\text{london}),(\text{frankreich},\text{paris}),(\text{spanien},\text{madrid}) \}$

\paragraph{Rechenregeln} 
\begin{itemize}
\item $\neg \forall x \; F \equiv \exists x \neg F$
\item $\neg \exists x \; F \equiv \forall x \neg F$
\item Für $Q \in \{ \forall, \exists \}$ und $\circ \in \{ \wedge, \vee \}$ gilt $(Q x F ) \circ G \equiv Q ( F \circ G )$, falls $x$ in $G$ nicht frei vorkommt.
\item $\forall x F \wedge \forall x G \equiv \forall x (F \wedge G)$
\item $\exists x F \vee \exists x G \equiv \exists x (F \vee G )$
\end{itemize}

\paragraph{Definition} Eine Aussage $F$ ist in bereinigter Pränexform wenn $F= Q_1 y_1 \dots Q_n y_n G$, wobei $Q_i \in \{ \forall , \exists \}$ und $G$ keine Quantoren enthält und $y_1,\dots,y_n$ paarweise verschieden sind.

Für jede Aussage gibt es eine äquivalente Formel in bereinigter Pränexform.

\subparagraph{Beispiel} \begin{align*} 
&\forall x ( \text{land} (x) \rightarrow \exists y \text{hauptstadt} (x,y)) &\equiv \\ 
&\forall x (\neg \text{land}(x) \vee \exists y \text{hauptstadt} (x,y)) &\equiv \\
&\forall x (\exists y \text{hauptstadt}(x,y) \vee \neg \text{land} (x)) &\equiv\\
&\forall x \exists y (\neq \text{land}(x) \vee \text{haupstadt}(x,y) ) &\equiv\\
&\forall x \exists y (\text{land} (x) \rightarrow \text{hauptstadt} (x,y))\\
\end{align*}

\begin{align*}
&\neg (\forall x(\text{vogel}(x) \rightarrow \text{fliegt} (x) ) ) &\equiv \\
&\neg (\forall x(\neg\text{vogel}(x) \vee \text{fliegt} (x) )) &\equiv \\
&\exists x \neg (\neg \text{vogel} (x) \vee \text{fliegt} (x) ) &\equiv \\
&\exists x (\text{vogel} (x) \wedge \neg \text{fliegt} (x))\\
\end{align*}

\paragraph{Definition}
\begin{itemize}
\item Ein \underline{Literal} ist eine Atomformel oder eine negierte Atomformel.
\item Eine Formel heißt \underline{Hornklausel}, wenn sie eine $\vee$-Verknüpfung von Literalen ist, von denen höchstens eins positiv ist.
\item Eine Hornformel ist eine $\wedge$-Verknüpfung von Hornklauseln.
\end{itemize}

\subparagraph{Beispiel}
\begin{itemize}
\item Literale: $P(x,y), \neg P(x,y), Q(zx)$\\
\item Hornklausel: $\neg P(x,y) \vee \neg Q(zx), \neg S(x) \vee T(y)$
\end{itemize}

\subparagraph{Wichtiger Spezialfall} Hornklausel mit genau einem positivem Literal.
\begin{itemize}
\item Ein Literal: Fakt, z.B. $P(x,y)$.
\item Mindestens zwei Literale: Dies lässt sich als Implikation darstellen.\\*
$ \neg A_1 \vee \dots \vee \neg A_{n-1} \vee A_n \equiv \neg (A_1 \wedge \dots \wedge A_{n-1} ) \vee A_n \equiv A_1 \wedge \dots \wedge A_{n-1} \rightarrow A_n$.
\end{itemize}

Mit Hilfe der Hornklausel lassen sich Regeln formulieren z.B.
\begin{align*}
&\forall x (\text{katze} (x) \rightarrow \text{haustier} (x)) &,\\
&\forall x (\text{hund} (x) \rightarrow \text{haustier} (x))&,\\
&\text{katze}(\text{reni})\\
\end{align*}
Wenn diese Hornklausel mit $\wedge$ verknüpft werden, erhalten wir eine Hornformel. Diese Hornformel kann als Wissensbasis eines Expertensystems betrachtet werden.

Grundlegender Aufbau eines Expertensystems
%TODO 2015-03-23T12:03 (Wissensbasis Inferenzsystem Anfrage Antwort

\section{Prolog-Programmierung}
\paragraph{Prolog} \underline{Pro}gramming in \underline{log}ic. In den 70ern und 80er Jahren als KI-Sprache entwickelt.

Prolog-Programme sind im Wesentlichen Hornformeln, bei denen alle Variablen allquantisiert sind und die in bereinigter Pränexform vorliegen.

Beispiel 
\begin{lstlisting}[numbers=left, tabsize=4, language=Prolog]
katze(reni).
haustier(X) :-katze(X).
\end{lstlisting}
Dieses Prolog-Programm stellt die Hornformel:
$\text{katze}(\text{reni}) \wedge \forall x(\text{katze}(x) \rightarrow \text{haustier}(x))$ dar.\\
Anfrage an Prolog:
\begin{lstlisting}[numbers=left, tabsize=4, language=Prolog]
? - haustier(reni).
true
\end{lstlisting}

\subsection{Syntax}
\paragraph{Prädikat} Wort in Kleinbuchstaben (außerhalb einer Klammer).
\paragraph{Konstante} Argument in Kleinbuchstaben.
\paragraph{Variablen} Wort, das mit Großbuchstaben beginnt.\\
"`$\leftarrow :$"' : "`$:-$"' (wird impliziert von)\\*
"`$\wedge$"' : "`,"'

Jede Klausel wird mit "`,"' abgeschlossen. Das Programm ist eine Menge von Klauseln, die implizit mit "`$\wedge$"' verknüpft sind (Hornformel).\\
$\vee$-Verknüpfung: Die Formel $A \vee B \rightarrow C$ ist wegen $A \vee B \rightarrow C \equiv \neq (A \vee B) \vee C \equiv (\neq A \wedge \neg B) \vee C$ keine Hornklausel. Da jedoch:
$A \vee B \rightarrow C \equiv (\neg A \vee C) \wedge (\neg B \vee C) \equiv (A \rightarrow C) \wedge (B \rightarrow C)$
eine Hornformel ist, lässt sich $A\vee B \rightarrow C$ in Prolog darstellen.

$\vee$-Operator: "`;"'\\*
Beispiel $\text{haustier}(X) :- \text{katze} (X); \text{hund}(x)$















\end{document}